\documentclass[preview]{standalone}
\usepackage{tikz}


\usetikzlibrary{decorations.pathreplacing}


\begin{document}

\color{white}

\begin{center}
    \begin{tikzpicture}[scale=1.5]
      % Draw ceiling
      \draw[thick,white] (-2.5,3) -- (2.5,3);
  
      % Draw pivot point
      \fill[white] (0,3) circle (0.1);
  
      % Calculate pendulum length (3 units)
      \def\pendulumLength{3}
  
      % Draw pendulum at equilibrium position (vertical)
      \draw[thick,white] (0,3) -- (0,{3-\pendulumLength});
      \fill[white] (0,{3-\pendulumLength}) circle (0.25);
  
      % Draw left swing position (30 degrees)
      \draw[thick,dashed,white] (0,3) --
      ({-\pendulumLength*sin(30)},{3-\pendulumLength*cos(30)});
      \fill[white,opacity=0.3]
      ({-\pendulumLength*sin(30)},{3-\pendulumLength*cos(30)}) circle (0.25);
  
      % Draw right swing position (30 degrees)
      \draw[thick,dashed,white] (0,3) --
      ({\pendulumLength*sin(30)},{3-\pendulumLength*cos(30)});
      \fill[white,opacity=0.3]
      ({\pendulumLength*sin(30)},{3-\pendulumLength*cos(30)}) circle (0.25);
  
      % Draw connecting arc for the path
      \draw[thick,dotted,white]
      ({-\pendulumLength*sin(30)},{3-\pendulumLength*cos(30)}) arc
      (-150:-30:1.75 and 0.85);
  
      % Draw angle theta (to right swing position)
      \draw[->,thick,white] (0,2.4) arc (-90:-45:0.4);
      \node[white] at (0.2,2.2) {$\theta$};
  
      % Label theta max at the right swing position
      \node[white] at (1.5,0) {$\theta_{\mathrm{max}}$};
  
      % Label negative theta max at the left swing position
      \node[white] at (-1.5,0) {$-\theta_{\mathrm{max}}$};
  
      % Label equilibrium position
      \node[white] at (0,-0.5) {$\theta = 0$};
  
      % Oblique overbrace representing L for the right pendulum
      \draw[decorate,decoration={brace,amplitude=10pt},white] (0,3) --
      ({\pendulumLength*sin(30)},{3-\pendulumLength*cos(30)})
      node[midway,above,xshift=17pt,white] {$L$};
    \end{tikzpicture}
  \end{center}
  
\end{document}